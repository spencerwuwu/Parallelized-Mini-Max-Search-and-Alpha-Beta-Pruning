% This is "sig-alternate.tex" V2.1 April 2013
% This file should be compiled with V2.5 of "sig-alternate.cls" May 2012
%
% This example file demonstrates the use of the 'sig-alternate.cls'
% V2.5 LaTeX2e document class file. It is for those submitting
% articles to ACM Conference Proceedings WHO DO NOT WISH TO
% STRICTLY ADHERE TO THE SIGS (PUBS-BOARD-ENDORSED) STYLE.
% The 'sig-alternate.cls' file will produce a similar-looking,
% albeit, 'tighter' paper resulting in, invariably, fewer pages.
%
% ----------------------------------------------------------------------------------------------------------------
% This .tex file (and associated .cls V2.5) produces:
%       1) The Permission Statement
%       2) The Conference (location) Info information
%       3) The Copyright Line with ACM data
%       4) NO page numbers
%
% as against the acm_proc_article-sp.cls file which
% DOES NOT produce 1) thru' 3) above.
%
% Using 'sig-alternate.cls' you have control, however, from within
% the source .tex file, over both the CopyrightYear
% (defaulted to 200X) and the ACM Copyright Data
% (defaulted to X-XXXXX-XX-X/XX/XX).
% e.g.
% \CopyrightYear{2007} will cause 2007 to appear in the copyright line.
% \crdata{0-12345-67-8/90/12} will cause 0-12345-67-8/90/12 to appear in the copyright line.
%
% ---------------------------------------------------------------------------------------------------------------
% This .tex source is an example which *does* use
% the .bib file (from which the .bbl file % is produced).
% REMEMBER HOWEVER: After having produced the .bbl file,
% and prior to final submission, you *NEED* to 'insert'
% your .bbl file into your source .tex file so as to provide
% ONE 'self-contained' source file.
%
% ================= IF YOU HAVE QUESTIONS =======================
% Questions regarding the SIGS styles, SIGS policies and
% procedures, Conferences etc. should be sent to
% Adrienne Griscti (griscti@acm.org)
%
% Technical questions _only_ to
% Gerald Murray (murray@hq.acm.org)
% ===============================================================
%
% For tracking purposes - this is V2.0 - May 2012

\documentclass{sig-alternate-05-2015}

\usepackage{hyperref}

\begin{document}

% Copyright
%\setcopyright{acmcopyright}
%\setcopyright{acmlicensed}
%\setcopyright{rightsretained}
%\setcopyright{usgov}
%\setcopyright{usgovmixed}
%\setcopyright{cagov}
%\setcopyright{cagovmixed}

%
% --- Author Metadata here ---
\CopyrightYear{2016} % Allows default copyright year (20XX) to be over-ridden - IF NEED BE.
%\crdata{0-12345-67-8/90/01}  % Allows default copyright data (0-89791-88-6/97/05) to be over-ridden - IF NEED BE.
% --- End of Author Metadata ---

\title{Parallelized Mini-Max Search and Alpha-Beta Pruning}
%
% You need the command \numberofauthors to handle the 'placement
% and alignment' of the authors beneath the title.
%
% For aesthetic reasons, we recommend 'three authors at a time'
% i.e. three 'name/affiliation blocks' be placed beneath the title.
%
% NOTE: You are NOT restricted in how many 'rows' of
% "name/affiliations" may appear. We just ask that you restrict
% the number of 'columns' to three.
%
% Because of the available 'opening page real-estate'
% we ask you to refrain from putting more than six authors
% (two rows with three columns) beneath the article title.
% More than six makes the first-page appear very cluttered indeed.
%
% Use the \alignauthor commands to handle the names
% and affiliations for an 'aesthetic maximum' of six authors.
% Add names, affiliations, addresses for
% the seventh etc. author(s) as the argument for the
% \additionalauthors command.
% These 'additional authors' will be output/set for you
% without further effort on your part as the last section in
% the body of your article BEFORE References or any Appendices.

\numberofauthors{3} %  in this sample file, there are a *total*
% of EIGHT authors. SIX appear on the 'first-page' (for formatting
% reasons) and the remaining two appear in the \additionalauthors section.
%
\author{
% You can go ahead and credit any number of authors here,
% e.g. one 'row of three' or two rows (consisting of one row of three
% and a second row of one, two or three).
%
% The command \alignauthor (no curly braces needed) should
% precede each author name, affiliation/snail-mail address and
% e-mail address. Additionally, tag each line of
% affiliation/address with \affaddr, and tag the
% e-mail address with \email.
%
% 1st. author
\alignauthor
Wei-Cheng Wu\\
       \affaddr{Electrical Engineering}\\
       \affaddr{and}\\
       \affaddr{Computer Science}\\
       \affaddr{Honors Program,}\\
       \affaddr{NCTU}\\
       \email{spencerwu85@gmail.com}
% 2nd. author
\alignauthor
Tsung-en Hsiao\\
       \affaddr{Electrical Engineering}\\
       \affaddr{and}\\
       \affaddr{Computer Science}\\
       \affaddr{Honors Program,}\\
       \affaddr{NCTU}\\
       \email{tsn@cs.nctu.edu.tw}
% 3rd. author
\alignauthor
Yi-Fan Wu\\
       \affaddr{Department of}\\
       \affaddr{Computer Science,}\\
       \affaddr{NCTU}\\
       \email{jameswu1212@gmail.com}
}
% There's nothing stopping you putting the seventh, eighth, etc.
% author on the opening page (as the 'third row') but we ask,
% for aesthetic reasons that you place these 'additional authors'
% in the \additional authors block, viz.
\date{\today}
% Just remember to make sure that the TOTAL number of authors
% is the number that will appear on the first page PLUS the
% number that will appear in the \additionalauthors section.

\maketitle
\keywords{Parallel, Alpha-beta Search, Mini-max Search, AI, Game, Learning}

\section{Introduction}
\href{https://en.wikipedia.org/wiki/Minimax}{Mini-Max search} with \href{https://en.wikipedia.org/wiki/Minimax}{Alpha-Beta Pruning} is a common method to implement game AI such as Chess, Go, Renju etc. We can utilize parallel programming techniques on those algorithms. We will implement several games with different parameters, and show how parallel programming speed up the calculation.
\section{Statement of the Problem}
We want to implement the Mini-Max Search and Mini-Max search with Alpha-beta Pruning in both parallel way and normal way. We can evaluate how parallelization affects game AI systems.
\section{Proposed Approches}
Mini-Max search can be easily parallelized by spliting the tree from the root, make the children of the root be the root node of the threads created. We are going to utilize both CPU and GPU. We will compare all the results in a graph. 
\section{Language Selection}
In the CPU part, we want to use OpenMP. OpenMP is easier for us to convert our squential code into a parallel program. In the GPU part, we want to use OpenCL for our implementation since it is more portable and a open standard.
\section{Related Work}
Several works have been done on this subject. Includes the utilization of GPU and CPU. Kamil Rocki et al of University of Tokyo implement mini-max tree searching on GPU. Brian Greskamp of UIUC shows how to parallizing chess program by PVSPlit algorithm.   
\section{Expected Results}
We expect to spend less time through parrallized mini-max than alpha-beta pruning to run the same data. Also, we should have a better performance with more threads created and more CPU cores used in our model.
\section{Time Table}
\begin{itemize}
    \item November, 2016 - Finish the sequencial program of our model.
    \item December, 2016 - Convert the sdquencial code into parrallel program. Tsung-en Hsiao will handle the GPU model. The openMP model will be done by Yi-Fan and Wei-Cheng Wu. 
    \item January, 2017 - Prepare for the final presentation.
\end{itemize}
\section{References}
\begin{enumerate}
    \item \href{http://olab.is.s.u-tokyo.ac.jp/~kamil.rocki/rocki_ppam09.pdf}{Parallel Minimax Tree Searching on GPU} by Kamil Rocki and Reiji Suda, 2010.
    \item \href{http://iacoma.cs.uiuc.edu/~greskamp/pdfs/412.pdf}{Parallelizing a Simple Chess Program} by Brian Greskamp, 2003.
\end{enumerate}
\end{document}
